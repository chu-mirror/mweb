% Citation Style, I adopt MLA here
\newcount\refcount
\refcount=0
\def\ref#1#2#3{\global\advance\refcount by 1
\frenchspacing % follow suggestion of The-TeXbook
\item{[\the\refcount]} #1. {\sl #2}. #3 \smallskip}

\beginsection Program and Process\par

Softwares, from a user's perspective, they are something running, handling some input,
producing some output, performing some tasks, but where are they?  You might say that they are in
smartphones, or if you have some backgound knowledge, any electronic device.  Well, but,
suppose that I have a software named {\tt AplusB}, it only do a simple job, you tell it two numbers,
and it reply to you with a sum of those numbers.  It's a software, you have no difficulty
in imagining installing it to your treasured iphone of newest generation.  Now let's assume that a people do the
same job for you, namely summing up two numbers you give, then you have a software installed on
that people literally, congratulations.  So what are softwares, it seems like they can be anywhere,
they are something untouchable, like spirits.
But as a creation, softwares comes from human's intelligent activities, and are not different
to any other creations, paintings, novels, or buildings.  What I want to say is that,
they are understandable to human beings, and as a consequence, they are descriptable.
Back to the first sentence, what can you say about a software? It's not hard to give a general
description, like I have done for my simple {\tt AplusB}.  If your description can result
in some actions doing what you want, the accumulated actions, we call it software,
and your description, is exactly a program, and the activity you giving a description,
is programming.

Another question shows up immediately, how can we even get all these actions by only a description,
it's not at all arcane, there are some entities able to understand our description
and do what we want them to do.  If you are smart enough, you may notice that the entities are
just other softwares, the behaviour of that people to compute {\tt AplusB} is a software,
the behaviour of electronic devices to go through bits to run {\tt AplusB} is a software.
It's kind of confussing to continue using software as the name for your new impression of these
pretty creatures, let's adopt the term from {\sl SICP}, {\it process} [1].

Conclusion and introducing of a new term, the relation between program and software,
now program and process, is {\it interpretation}.  A program is interpreted to a process by another process.

\beginsection Literate Programming, MWEB's approach\par

From the previous section, we know that programming is all about giving description to process you want
to create, and apparently the description you give is not for human's reading but another process's interpretating.
{\it Literate~Programming} is a programming paradigm against this nature of programming [2], introduced by
Donald Knuth, it regards description of a process as both a program to be interpreted and a document written
for human beings.  I will not stress the benefits of it here, the paper by Donald~Knuth gives
a perfect explanation for it.

In general terms, {\tt MWEB} is yet another literate programming tool, it improved the classic paradigm
by introducing a type system.  Consider the final process a program interpreted to, it's not only
influenced by the program but also the process doing interpret.  For example, you have a C program,
you can compile and link it to build a machine program interpreted by particular combination of operating system
and hardware to result in a normal process, but also, you can pass an argument {\tt -g} to the compiler
to result in a different process able to be debugged.  Same program, but different processes.
{\tt MWEB} bring in the causing of different processes to consideration.  This is
the key difference between classic approach and {\tt MWEB}'s approach, {\it classic literate programming
regards a program as the object being described, {\tt MWEB} regards a process as the object being described}.

\beginsection MWEB's type system\par

\beginsection Bibliography\par

\ref {Abelson, Sussman, and Sussman} {Structure and Interpretation of Computer Programs}{MIT Press, 1984}
\ref {Donald E. Knuth} {Literate Programming} {1984}

\bye
